\documentclass[a4paper,12pt]{article}
\usepackage{graphicx}
\usepackage{enumitem}
\usepackage{titlesec}
\usepackage{geometry}

% Title format
\titleformat{\section}{\large\bfseries}{}{0em}{}

% Page margins
\geometry{left=3cm, right=3cm, top=3cm, bottom=3cm}

\title{Master Thesis Proposal Template with Time Management Strategy}

\author{Jonas Urban}

\date{\today}

\begin{document}

\maketitle

\section*{Instructions}
This template provides a structured outline for your master thesis proposal and includes a time management exercise. Complete each section with brief responses (1–2 sentences or bullet points) to save time. After completing, push your updates to GitHub as practice.

\section{1. Summary}
We investigate the indoor air pollution with low-cost sensors to improve the exposure management and improve housing construction\\

\section{2. Research Background}
Many studies on outdoor air pollution and with many different measuring devices and calibration methods. No real standardization in any of these processes. \\

\section{3. Research Objectives (incl. Research Questions)}
How can we improve the measurements by low cost air pollution sensors using machine learning algorithms. 

\section{4. Methods and Data}
A measurement campaign indoors (office) with many different devices for many different substances together with reference devices.\\

\section{5. Timeline and Milestones (incl. Gantt Chart)}
4 weeks literature researh\\
4 weeks modelling \\
4 weeks indtroduction and methods writing \\
4 weeks nice figures +stattistics \\
4 weeks results and conclusion writing \\
4 weeks feedback polishing \\
2 weeks buffer \\

\section{Exercise: Time Management Strategy}
Use this section to outline your weekly planning, prioritisation, and reflection. Keep responses short and focused.

\subsection*{Weekly Planning}
\textbf{Prompt:} List 1–2 key tasks you need to complete this week.\\
\textbf{Answer:}
\begin{itemize}
    \item finish data and methods chapter
    \item include appropriate sources for literature review/background
\end{itemize}

\subsection*{Task Prioritisation}
 Categorise a few tasks as: 1. Urgent and Important, 2. Important but Not Urgent, 3. Urgent but Not Important, 4. Not Urgent and Not Important.\\
\textbf{Answer:}
\begin{itemize}
    \item Urgent and Important: finish data and methods chapter
    \item Important but Not Urgent: appropriate sources for literature review/background
    \item Urgent but Not Important: 
    \item Not Urgent and Not Important: 
\end{itemize}

\subsection*{Reflection and Adjustment}
Briefly reflect on your progress this week. What went well? What challenges did you face?\\
\textbf{Answer: Went well} 

\noindent
\textbf{Adjustment for Next Week: None} 


\end{document}
